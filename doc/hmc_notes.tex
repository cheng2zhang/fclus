%\documentclass[reprint,superscriptaddress]{revtex4-1}
\documentclass[aip,jcp,preprint,notitlepage, superscriptaddress]{revtex4-1}
\usepackage{amsmath}
\usepackage{bm}
\usepackage[usenames,dvipsnames]{xcolor}
\usepackage{tikz}
\usepackage{hyperref}

\hypersetup{
    colorlinks,
    linkcolor={red!30!black},
    citecolor={blue!50!black},
    urlcolor={blue!80!black}
}

\begin{document}


\newcommand{\vct}[1]{\bm{\mathrm{#1}}}
\newcommand{\vx}{\vct{x}}
\newcommand{\vy}{\vct{y}}
\newcommand{\Z}{\mathcal{Z}}
\newcommand{\E}{\mathcal{E}}
\newcommand{\Ham}{\mathcal{H}}
\newcommand{\W}{\mathcal{W}}
\newcommand{\A}{\mathcal{A}}

% annotation macros
\newcommand{\repl}[2]{{\color{gray} [#1] }{\color{blue} #2}}
\newcommand{\add}[1]{{\color{blue} #1}}
\newcommand{\del}[1]{{\color{gray} [#1]}}
\newcommand{\note}[1]{{\color{OliveGreen}\small [\textbf{Comment.} #1]}}

\newcommand{\hl}[1]{{\color{red} #1}}

\tableofcontents


\section{MD integrators}


\subsection{Overview}



Commonly used MD integrators are symplectic and reversible.
%
In each MD step,
the integrator changes the coordinates and momenta slight.
%
The small change can be considered as a map
in the phase space of $(\vct q, \vct p)$
(which is actually a \emph{symplectic manifold}).
%
A symplectic transformation is a map
from $(\vct q(t), \vct p(t))$
to $\vct q(t + \Delta t), \vct p(t + \Delta t)$
that conserves the total area projections
onto the $n$ individual sub phase planes $(q_i, p_i)$.
%
We can show that a symplectic transformation
preserves the form of Hamilton's equations,
which means that it is equivalent to
a canonical transformation in classical mechanics.


The natural time evolution of a Hamiltonian,
that is, the map from $(\vct q(t), \vct p(t))$
to $(\vct q(t+\Delta t), \vct p(t+\Delta t))$,
can also be considered as a canonical, hence symplectic, transformation.
%
The converse problem is more interesting,
given a canonical transformation,
can we find the Hamiltonian that that generates the flow?
%
For the canonical transformation associated to an MD integrator,
this Hamiltonian is called the ``shadow''
(or modified) Hamiltonian,
and it remains a constant along the trajectory
(within the limit of floating-point).
%
This properties is extremely useful,
because it gives a measure of long time stability.



\subsection{Review of canonical transformations}



There are several equivalent definition of canonical transformations.
%
\begin{enumerate}

\item
It preserves the form of Hamilton equations
for \emph{any} Hamiltonian $H$,
i.e.,
$$
\begin{aligned}
\dot {\mathbf q} &=  \frac{ \partial H }{ \partial \vct p }, \\
\dot {\mathbf p} &= -\frac{ \partial H }{ \partial \vct q }.
\end{aligned}
$$

This condition is equivalent to the symplectic conditoin,
which is also equivalent to saying that the Poisson brackets
$$
\{ f, g \}_{\vct q, \vct p} =
\sum_i
\frac{ \partial f } {\partial q_i}  \frac{ \partial g } { \partial p_i }
-
\frac{ \partial f } {\partial p_i}  \frac{ \partial g } { \partial q_i }
$$
of any two function $f$ and $g$ are the same
when evaluated for $(\vct q(t), \vct p(t))$
or for $(\vct q(t+\Delta t), \vct p(t+\Delta t))$.

\item
It preserves the 2-form
$\omega^2 = d\vct p \wedge d\vct q = \sum_i dp_i \wedge dq_i$.
%
This result can be interpreted geometrically.
%
Two tangent vectors
$(d\vct q_1, d\vct p_1)$ and $(d\vct q_2, d\vct p_2)$
in the $(\vct q, \vct p)$ phase space,
spans a parallelogram.
%
The total oriented area of the projections
of the parallelogram onto the $n$ $(q_i, p_i)$ planes,
is a constant.
%
This area is computed by $d\vct p \wedge d\vct q$.

This property implies the better-known Liouville volume theorem,
that is, the volume element is a conserved by the Hamiltonian flow.
%
Note that the conservation of total projected area
implies the conservation of the volume element
(or $dp_1 \wedge dq_1 \wedge \cdots \wedge dp_n \wedge dq_n$,
which can be shown to be proportional to
the $n$th exterior power of $\omega^2$),
but the converse is not true.


\item
A canonical transformation $(\vct p, \vct q) \rightarrow (\vct P, \vct Q)$
can be generate by a generating functions.
%
There are four classes of generating functions.
%
Take the first class $F_1(\vct q, \vct Q)$ for example,
$$
dF_1 = \vct p \, d\vct q - \vct P \, d\vct Q.
$$
If the canonical transformation is a natural time evolution,
then $F_1 = -\Delta S = -\int_{t}^{t + \Delta t} L \, d\tau$
is the negative difference of the action $S$,
which is an integral of the Lagrangian.


\end{enumerate}



\subsection{Generating functions}


Here, we will try to find the generating functions
of the leapfrog and velocity Verlet integrators.
%
These exercises will 1) show explicitly that
the MD integrators indeed transform coordinates and momenta
canonically, or symplectically;
2) gives the $F_1 = -\Delta S$ the action.
%
We shall use scalar notions below for convenience,
the vector versions are straightforward.



\subsubsection{Coordinate update step}


Let us start with something simple.  For a coordinate update step, we have
%
\begin{equation}
\begin{aligned}
Q &= q + \frac{p}{m} \, \Delta t, \\
P &= p.
\end{aligned}
\label{eq:qstep}
\end{equation}
%
The generating function $F_1$ satisfies
$$
dF_1 = p \, dq - P \, dQ = p \, dq - d (P \, Q) + Q \, dP.
$$
or, more conveniently, by switching to the second-class generating function
$$
dF_2 = d (F_1 + P \,Q) = p \, dq + Q \, dP,
$$
where $F_2$ is a function of $q$ and $P$.

Then, using \eqref{eq:qstep}, we get
$$
dF_2
= P dq + \left(q + \frac{P}{m}\Delta t\right) dP.
$$
This equation is satisfied by
$$
F_2 = P \, q + \frac{P^2}{2 \, m} \, \Delta t,
$$
which means the corresponding $F_1$ is given by
$$
F_1 = F_2 - P Q = -\frac{P^2}{2 \, m} \, \Delta t.
$$
Since this quantity is the negative of $\Delta S$,
we find that the equivalent Lagrangian during $\Delta t$ is
the kinetic energy $L = K = \frac{ P^2 }{2 \, m}$.



\subsubsection{Momentum update step}


For a momentum update step, we have
%
\begin{equation}
\begin{aligned}
Q &= q, \\
P &= p - \frac{ \partial U(Q) } { \partial Q } \Delta t.
\end{aligned}
\end{equation}
%
Let us find the equivalent $F_2$.
%
$$
dF_2 = p \, dq + Q \, dP
= \left( P + \frac{\partial U(q) }{ \partial q} \, \Delta t\right) \, dq + q \, dP.
$$
This equation is satisfied by $F_2 = P \, q + U(q) \Delta t$,
and correspondingly
$$
F_1 = F_2 - P\, Q = U(q) \Delta t,
$$
Since this quantity is $-\Delta S$,
the equivalent Lagrangian during the update is
the negative potential energy $L = -U(q)$.


\subsubsection{Leapfrog integrator (velocity first)}


In a velocity-first leapfrog step,
%
\begin{equation}
\begin{aligned}
P &= p - \frac{ \partial U(q) } { \partial q } \Delta t, \\
Q &= q + \frac{P}{m} \Delta t.
\end{aligned}
\end{equation}
%
Let us find the equivalent $F_2$.
%
$$
dF_2 = p \, dq + Q \, dP
= \left( P + \frac{\partial U(q) }{ \partial q} \, \Delta t\right) \, dq
 + \left( q + \frac{ P }{m} \Delta t \right) \, dP.
$$
This equation is satisfied by
$$
F_2 = P \, q + \left[ U(q) + \frac{P^2}{2m} \right] \Delta t,
$$
and correspondingly
$$
F_1 = F_2 - P\, Q = \left[ U(q) - \frac{P^2}{2m} \right] \Delta t,
$$
This expression resembles the true Lagrangian $L = K - U$.
The kinetic and potential energies are, however, evaluated at two different points.




\subsubsection{Leapfrog integrator (position first)}


In a position-first leapfrog step,
%
\begin{equation}
\begin{aligned}
Q &= q + \frac{p}{m} \Delta t, \\
P &= p - \frac{ \partial U(Q) } { \partial Q } \Delta t.
\end{aligned}
\end{equation}
%
For this, let us use a different generating function $F_4$.
Recall that
$$
dF_1 =  p \, dq - P \, dQ = d (p \, q) - q \, dp - P \, dQ.
$$
So
$$
dF_4 = d(F_1 - p \, q) = -q \, dp - P \, dQ
$$
%
In our case
$$
dF_4
=
 - \left( Q - \frac{ p }{ m } \, \Delta t\right) \, dp
 - \left( p - \frac{ \partial U(Q) } { \partial Q } \Delta t \right) \, dQ.
$$
This equation is satisfied by
$$
F_4 = -Q \, p + \left[ U(Q) + \frac{p^2}{2m} \right] \Delta t,
$$
and correspondingly
$$
F_1 = F_4 + p \, q = \left[ U(q) - \frac{p^2}{2m} \right] \Delta t.
$$


\subsection{Velocity-Verlet integrator}


An velocity Verlet step is equivalent to
a velocity-first leapfrog step of time $\Delta/2$
followed by
a position-first leapfrog step of time $\Delta/2$.
%
\begin{equation}
\begin{aligned}
P     &= p_1 - \frac{ \partial U(q_1) } { \partial q_1 } \frac{ \Delta t }{ 2 }, \\
q_2   &= q_1 + \frac{P}{m} \Delta t, \\
p_2   &= P - \frac{ \partial U(q_2) } { \partial q_2 } \frac{ \Delta t } { 2 }.
\end{aligned}
\end{equation}
%
We can then simply combine the generating functions
of the two types of velocity Verlet steps.
$$
\begin{aligned}
F_1
&=
\left[
\frac{ U(q_1) + U(q_2) } { 2 }
- \frac{ P^2 }{ 2 m }
\right] \, \Delta t \\
&=
\frac{ U(q_1) + U(q_2) } { 2 } \, \Delta t
- \frac{1}{2} \, m \, \frac{ (q_2 - q_1)^2 }{ \Delta t }.
\end{aligned}
$$



\section{Wang-Landau algorithm}



\subsection{Inverse time prescription}

We now turn to the issue of optimally assign the updating magnitude.

Asymptotically, we should have
\begin{equation}
\ln f = \frac{C}{t},
\label{eq:lnfinvt}
\end{equation}
where $t$ is to be interpreted as the number of simulation steps.
%
The effect of the optimal $\ln f$ should be independent of the correlation time $\tau$.
That is, for two systems with different correlation times, $\tau_1$ and $\tau_2$,
the updating behavior should look similarly
when their time scales are divided by the respective correlation times.
%
Since the number of steps that a system has to stay within a bin before making a transition
should be proportional to the correlation time,
the amount of $\ln f$ accumulated within this period
is proportional to $\ln f \tau$.
%
This value should be the same among systems with different correlation times.
%
Thus,
\begin{equation}
  (\ln f) \, \tau = \mathcal F(t/\tau).
\label{eq:lnf_general}
\end{equation}


We note that the only choice
such that $\ln f$ does not depend on $\tau$ explicitly
is $\mathcal F(x) = C/x$, which is \eqref{eq:lnfinvt}.
Below we shall show that this is indeed the asymptotical form.


We now try to determine the asymptotical functional form of
$\mathcal F(t)$ for long times $t$.
%
Without loss of generality,
we can assume perfect sampling, with $\tau = 1$.
%
Consider a long period of $t$, during which
the bin $i$ is visited by $n_i$.  If the number of bins is large,
$n_i$ satisfies a Poisson distribution, and
$$
\overline{ n_i }
= \mathrm{var}( n_i )
= \frac{ w_i }{ w_\mathrm{tot} } \, t,
$$
where $w_i$ is the weight of bin $i$ such that the minimum among all bins is $1.0$
and $\sum_i w_i = w_\mathrm{tot}$).
The variance is defined as
$$
\mathrm{var}( n_i )
= \overline{ (n_i - \overline{ n_i } )^2 }.
$$


Now during this period, the bias potential at bin $i$
is increased by $\Delta V_i = n_i \, \ln f / w_i$.
It is readily seen that the mean
$$
\overline{ \Delta V_i }
=
\ln f \, \frac{ t } { w_\mathrm{tot} },
$$
which is a constant independent of the bin index.
%
This uniform shift of the bias potential is trivial,
and only the standard deviation affects the precision.
%
$$
\mathrm{var} (\Delta V_i)
=
\left( \frac{\ln f}{w_i} \right)^2 \mathrm{var}( n_i )
=
\left( \frac{\ln f}{w_i} \right)^2 \frac{w_i}{w_\mathrm{tot}} t.
$$
%
For long times, the systematic error of $V_i$ should have been damped out
by the updating procedure, so the above result also gives
the variance of the estimated bias potential
\begin{equation}
\mathrm{var} ( V_i )^\mathrm{new}
=
\mathrm{var} ( V_i )^\mathrm{old}
+
\left( \frac{\ln f}{w_i} \right)^2 \frac{w_i}{w_\mathrm{tot}} t.
\label{eq:Vi_lnf}
\end{equation}



For long times $t$, $\ln f$ is sufficiently small,
and the updating does not affect the correlation time $\tau$,
i.e., we are in the regime of equilibrium sampling.
%
Thus, the bias potential can be optimally found from inverting the observed histogram
$V_i = \ln n_i$, and the variance is
\begin{equation}
\mathrm{var} (V_i)
=
\frac{ \langle n_i^2 \rangle } { \langle n_i \rangle^2 }
=
\frac{ w_\mathrm{tot} } {w_i \, t}.
\label{eq:Vi_histogram}
\end{equation}
%
Thus the value given by \eqref{eq:Vi_lnf}
cannot be less than \eqref{eq:Vi_histogram},
and this yields
the optimal $\ln f$
\begin{equation}
  \ln f \ge \frac{ w_\mathrm{tot} } { t }.
\end{equation}

This is the $1/t$ prescription for a general distribution.
%
For a flat histogram, $w_\mathrm{tot} = N_b$
is the number of bins, and it recovers
the original recipe\cite{
belardinelli2007, belardinelli2008}.



\bibliography{simul}
\end{document}
