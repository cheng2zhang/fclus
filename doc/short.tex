\documentclass{article}
\usepackage{amsmath}



\newcommand{\vct}[1]{\mathbf{#1}}
\newcommand{\vx}{\vct{x}}
\newcommand{\vr}{\vct{r}}
\newcommand{\vv}{\vct{v}}
\newcommand{\Q}{\mathcal{Q}}



\begin{document}



% Aggregation problem.
%
In a multiple-component liquid mixture,
a homogeneously dispersed solute-like substance
can aggregate and form larger clusters
under certain temperature and concentration.
%
A challenge is to predict if this would occur
for a given condition.



% Free energy calculation.
%
The problem can be tackled
from the free energy perspective,
%
in which one computes
the amount of free energy gained or lost
in forming an aggregate from the dispersed state.
%
More generally,
a free energy calculation
aims at computing the free energy difference
between two well-characterized states $A$ and $B$,
which, in our case,
are the dispersed and aggregated states, respectively.


% Umbrella sampling.
%
A standard technique for free energy calculation is
umbrella sampling.
%
For molecular systems,
a molecular dynamics (MD) implementation
is usually more popular than
a Monte Carlo (MC) one,
and we shall focus on the former below.
%
First,
we need to define
from the molecular coordinates, $\vr$,
a reaction coordinate, $Q(\vr)$,
which assumes different values
at states $A$ and $B$.
%
The free energy $F(q)$
at given value of $q = Q(\vr)$
is given by
%
\begin{equation}
e^{-\beta F(q)}
=
Z^{-1}
\int
  e^{-\beta U_0(\vr)} \,
  \delta[Q(\vr) - q] \,
  d\vr,
\label{eq:freeenergydef}
\end{equation}
%
where,
$U_0(\vr)$
is the potential energy function,
and
$Z \equiv \int e^{-\beta U_0(\vr)} d\vr$.
%
In umbrella sampling,
we modify the potential energy as
%
\begin{equation}
U(\vr)
=
U_0(\vr)
+
V[Q(\vr)],
\end{equation}
%
with the bias potential
$V[Q(\vr)]$
depending on coordinates $\vr$
only through the reaction coordinate.
%
The bias potential
is updated throughout the simulation
to achieve a flat distribution along $Q$.
%
When this is accomplished,
one can show from Eq. \eqref{eq:freeenergydef}
that $V(q)$ can differ from $-F(q)$
by a constant shift only.



% Difficulty of umbrella sampling
% for a non-analytical order parameter.
%
However, the above scheme, when implemented in MD,
suffers from an obvious limitation.
%
The reaction coordinate, $Q$,
has to be a analytical function of the coordinates
in order to produce the force.
%
This restriction is inconvenient
for studying aggregation,
for some intuitive reaction coordinates
involving, e.g., the cluster size
are clearly not analytical.



% The method: free energy calculation along a non-analytical
% order parameter using molecular dynamics
%
Here we give a simple remedy
to enable free energy calculation
for a non-analytical reaction coordinate.
%
The idea is to use
the well-known technique of hybrid Monte Carlo (HMC)
to effect the influence of the bias potential $V[Q(\vr)]$.
%
Since HMC requires only energy but not the force,
the analyticity requirement is lifted.



% Detailed description of the method
%
In the simplest implementation, 
we repeat the following compound step until simulation ends.
%
Each compound step consists of three small steps:
%
\begin{enumerate}
  \item Execute $k$ MD steps with a stochastic thermostat,
        and let the system evolute
        from the state $(\vr, \vv)$ to $(\vr', \vv')$.

  \item Compute $\Delta V \equiv V[Q(\vr')] - V[Q(\vr)]$,
        and generate a uniform random number $r$ between $0$ and $1$.

  \item If $r < \min\{e^{-\beta \Delta V}, 1\}$,
        accept the new state $(\vr', \vv')$.
        Otherwise, change the current state to $(\vr, -\vv)$
        (notice the reversal of velocity).
\end{enumerate}



% On-the-fly update of the bias potential
%
The bias potential can be updated on-the-fly
using standard techniques.
%
The implementation using the Wang-Landau algorithm
is described below.
%
We first discretize the potential on a few bins along $Q$,
with the bias potential $V$ assuming a constant value $V_i$
within each bin $i$.
%
The initial $V_i$ can be uniformly zero.
%
At the end of each compound step,
we locate the bin $i$ containing the current $Q(\vr)$
and update the potential there as
%
\begin{equation}
V_i \rightarrow V_i + \ln f,
\end{equation}
%
where $\ln f$ is the updating magnitude.
%
No matter the magnitude of $\ln f$,
this updating scheme eventually leads to a flat distribution along $Q$,
although the accuracy of $V_i$ depends on $\ln f$.
%
Thus, the magnitude $\ln f$ should be reduced gradually.


% Example



\end{document}

